\clearpage
\makeatletter
\efloat@restorefloats
\makeatother


\begin{appendix}
\section{}
\hypertarget{the-bias-of-cohens-bmd-is-twice-as-large-as-the-bias-of-shiehs-bmd-when-population-variances-and-sample-sizes-are-equal-across-groups-mathematical-demonstration.}{%
\subsection{\texorpdfstring{The bias of Cohen's \(\bm{d}\) is twice as
large as the bias of Shieh's \(\bm{d}\) when population variances and
sample sizes are equal across groups: mathematical
demonstration.}{The bias of Cohen's \textbackslash bm\{d\} is twice as large as the bias of Shieh's \textbackslash bm\{d\} when population variances and sample sizes are equal across groups: mathematical demonstration.}}\label{the-bias-of-cohens-bmd-is-twice-as-large-as-the-bias-of-shiehs-bmd-when-population-variances-and-sample-sizes-are-equal-across-groups-mathematical-demonstration.}}

As mentioned in Table 1, the bias of Cohen's \(d\) is defined as
\begin{equation} 
Bias_{Cohen's \; d}= \delta_{Cohen} \times \left( \frac{\sqrt{\frac{df_{Student}}{2}} \times \Gamma{\left(\frac{df_{Student}-1}{2}\right)}}{\Gamma{\left( \frac{df_{Student}}{2}\right)}} -1 \right)
(\#eq:Cohenbias)
\end{equation} with \begin{equation*} 
\delta_{Cohen}=\frac{\mu_1-\mu_2}{\sqrt{\frac{(n_1-1)\times \sigma^2_1+(n_2-1)\times\sigma^2_2}{n_1+n_2-2}}}
(\#eq:Cohendelta)
\end{equation*} and \begin{equation*} 
df_{Student}=n_1+n_2-2
(\#eq:Cohendf)
\end{equation*}

As mentioned in Table 2, the bias of Shieh's \(d\) is defined as
\begin{equation} 
Bias_{Shieh's \; d}=\delta_{Shieh} \times \left( \frac{\sqrt{\frac{df_{Welch}}{2}} \times \Gamma{\left(\frac{df_{Welch}-1}{2}\right)}}{\Gamma{\left( \frac{df_{Welch}}{2}\right)}} -1 \right)
(\#eq:Shiehbias)
\end{equation} with \begin{equation*} 
\delta_{Shieh}=\frac{\mu_1-\mu_2}{\sqrt{\frac{\sigma^2_1}{n_1/N}+\frac{\sigma^2_2}{n_2/N}}} \quad (N=n_1+n_2)
(\#eq:Shiehdelta)
\end{equation*} and \begin{equation*} 
df_{Welch}=\frac{\left(\frac{\sigma^2_1}{n_1}+\frac{\sigma^2_2}{n_2} \right)^2}{\frac{(\sigma^2_1/n_1)^2}{n_1-1}+\frac{(\sigma^2_2/n_2)^2}{n_2-1}}
(\#eq:Welchdf)
\end{equation*}

When \(n_1=n_2=n\) and \(\sigma_1=\sigma_2=\sigma\), \(\delta_{Cohen}\)
is twice larger than \(\delta_{Shieh}\), as shown below in equations
\ref{eq:Cohendeltavarbalanced} and \ref{eq:Shiehdeltavarbalanced}:
\begin{equation} 
\delta_{Cohen}=\frac{\mu_1-\mu_2}{\sqrt{\frac{2(n-1)\sigma^2}{2(n-1)}}}=\bm{\frac{\mu_1-\mu_2}{\sigma}}
(\#eq:Cohendeltavarbalanced)
\end{equation} \begin{equation} 
\delta_{Shieh}=\frac{\mu_1-\mu_2}{\sqrt{2\left( \frac{\sigma^2}{n/(2n)}\right)}}=\bm{\frac{\mu_1-\mu_2}{2\sigma}} 
(\#eq:Shiehdeltavarbalanced)
\end{equation}\\
Moreover, degrees of freedom associated with Student's \emph{t}-test and
Welch's \emph{t}-test are identical, as shown below in equations
\ref{eq:Studentdfvarbalanced} and \ref{eq:Welchdfvarbalanced}:
\begin{equation} 
df_{Student}=\bm{2(n-1)} 
(\#eq:Studentdfvarbalanced)
\end{equation} \begin{equation} 
df_{Welch}=\frac{\left[2(\sigma^2/n)\right]^2}{\frac{2(\sigma^2/n)^2}{n-1}}= \bm{2(n-1)} 
(\#eq:Welchdfvarbalanced)
\end{equation}

Equations \ref{eq:Cohenbias} and \ref{eq:Shiehbias} can therefore be
redefined as follows: \begin{equation} 
Bias_{Cohen's \; d}=\frac{\mu_1-\mu_2}{\sigma} \times \left( \frac{\sqrt{n-1} \times \Gamma{\left(\frac{2n-3}{2}\right)}}{\Gamma{\left( n-1\right)}} -1 \right)
(\#eq:Cohenbiasvarbalanced)
\end{equation} \begin{equation} 
Bias_{Shieh's \; d}=\frac{\mu_1-\mu_2}{\bf 2\sigma} \times \left( \frac{\sqrt{n-1} \times \Gamma{\left(\frac{2n-3}{2}\right)}}{\Gamma{\left( n-1\right)}} -1 \right)
(\#eq:Shiehbiasvarbalanced)
\end{equation}

We can therefore conclude that the bias of Cohen's \(d\) is twice larger
than the bias of Shieh's \(d\).

\newpage

\hypertarget{the-variance-of-cohens-bmd-is-four-times-larger-than-the-bias-of-shiehs-bmd-when-population-variances-and-sample-sizes-are-equal-across-groups-mathematical-demonstration.}{%
\subsection{\texorpdfstring{The variance of Cohen's \(\bm{d}\) is four
times larger than the bias of Shieh's \(\bm{d}\) when population
variances and sample sizes are equal across groups: mathematical
demonstration.}{The variance of Cohen's \textbackslash bm\{d\} is four times larger than the bias of Shieh's \textbackslash bm\{d\} when population variances and sample sizes are equal across groups: mathematical demonstration.}}\label{the-variance-of-cohens-bmd-is-four-times-larger-than-the-bias-of-shiehs-bmd-when-population-variances-and-sample-sizes-are-equal-across-groups-mathematical-demonstration.}}

The variance of Cohen's \(d\) is defined in Table 1 as \begin{equation}
Var_{Cohen's \; d}=\frac{N\times df_{Student}}{n_1n_2 \times (df_{Student}-2)} + \delta^2_{Cohen} \left[ \frac{df_{Student}}{df_{Student}-2} - \left( \frac{\sqrt{\frac{df_{Student}}{2}} \times \Gamma{\left(\frac{df_{Student}-1}{2}\right)}}{\Gamma{\left( \frac{df_{Student}}{2}\right)}} \right)^2\right]
(\#eq:Cohenvar)
\end{equation} and the variance of Shieh's \(d\) is defined in Table 2
as \begin{equation}
Var_{Shieh's \; d}=\frac{df_{Welch}}{(df_{Welch}-2)N}  + \delta^2_{Shieh} \left[ \frac{df_{Welch}}{df_{Welch}-2} - \left( \frac{\sqrt{\frac{df_{Welch}}{2}} \times \Gamma{\left(\frac{df_{Welch}-1}{2}\right)}}{\Gamma{\left( \frac{df_{Welch}}{2}\right)}} \right)^2 \right]
(\#eq:Shiehvar)
\end{equation}

We have previously shown in equations \ref{eq:Studentdfvarbalanced} and
\ref{eq:Welchdfvarbalanced} that degrees of freedom associated with
Student's \emph{t}-test and Welch's \emph{t}-test equal \(2(n-1)\), when
\(n_1=n_2=n\) and \(\sigma_1=\sigma_2=\sigma\). As a consequence, the
first term of the addition in equation \ref{eq:Cohenvar} is 4 times
larger than the first term of the addition in equation
\ref{eq:Shiehvar}:
\[\frac{N\times df_{Student}}{n_1n_2 \times (df_{Student}-2)}=\frac{2n\times 2(n-1)}{n^2 \times (2n-4)} =\bm{\frac{4(n-1)}{n(2n-4)}} \]
\[\frac{df_{Welch}}{(df_{Welch}-2)N} = \frac{2(n-1)}{2n(2n-4)}= \bm{\frac{n-1}{n(2n-4)}}\]
We have also previously shown in equations
\ref{eq:Cohendeltavarbalanced} and \ref{eq:Shiehdeltavarbalanced} that
\(\delta_{Cohen}\) is twice larger than \(\delta_{Shieh}\) when
\(n_1=n_2=n\) and \(\sigma_1=\sigma_2=\sigma\) and, therefore,
\(\delta^2_{Cohen}\) is four times larger than \(\delta^2_{Shieh}\). As
a consequence, the second term of the addition in equation
\ref{eq:Cohenvar} is also 4 times larger than the second term of the
addition in equation \ref{eq:Shiehvar}. Because both terms of the
addition in equation \ref{eq:Cohenvar} are four times larger than those
in equation \ref{eq:Shiehvar}, we can conclude that the variance of
Cohen's \(d\) is four times larger than the variance of Shieh's \(d\).
\end{appendix}
